Das Verwenden von Rechenmaschinen um mathematische Operationen auszuführen ist seither eines der wichtigsten Fortschritte im Gebiet der Wissenschaft. Die ersten Ansätze dieser Rechner waren jedoch von den heutigen, leistungsstarken Computern weit entfernt.
\subsection{Der \glqq{}erste Computer\grqq{} der Welt}
Laut \cite{bruderer2011konrad} war Konrad Zuse (1910 - 1995) ein deutscher Bauingenieur, dessen Werke in \cite{konradz1z2} thematisiert werden. Er wird zudem als \glqq{}Vater des Computers\grqq{} bezeichnet. Jedoch steht bis heute nicht fest, wer der tatsächliche Erfinder des Computers war, da dies stark von der Definition eines solchen abhängt. Zuse war vor allem für seine revolutionären Erfindungen seit den dreißiger Jahren bekannt. Dazu gehört die sogenannte \textit{Z1}, ein mechanischer Apparat zum Durchführen arithmetischer Berechnungen. Diese entwickelte er von 1936 bis 1938 und funktionierte, statt mit elektrischem Strom, mithilfe von beweglichen Metallstiften, welche sich jeweils nur entweder vorwärts oder rückwärts entlang einer Schiene bewegen konnten.\\\\
Da sich hiermit keine Zahlen im konventionellen Dezimalsystem darstellen lassen, funktionierte die Z1 mithilfe des binären Zahlensystems. Im Gegensatz zum dezimalen Zahlensystem, bei dem für eine einzelne Stelle zehn verschiedene Ziffern zur Auswahl stehen, kann eine Stelle im Binärsystem lediglich einen von zwei möglichen Zuständen annehmen. Dies ist passend für die zwei möglichen Bewegungsrichtungen der Metallstäbe. Das binäre Zahlensystem wird in \fancyref{BinarySystem} genauer beschrieben. Auch längere und komplexere Ketten an Anweisungen konnten in binärer Schreibweise auf ein Lochband übertragen und anschließend ausgeführt werden. Gegenüber anderen, zu der Zeit entstandenen Rechnern, war deshalb die Z1 besonders flexibel.\\\\
Ebenso wie ein tatsächlicher Erfinder des Computers nicht feststeht, gibt es keine Erfindung, welche objektiv als \glqq{}erster Computer\grqq{} festgelegt werden kann. In \cite{konradz1z2} wird ebenso der an der Moore School of Electrical Engineering entstandene \glqq{}\textit{ENIAC}\grqq{} aufgelistet. Er wurde 1945 fertiggestellt und löste schon im Dezember desselben Jahres seine erste Rechenaufgabe. Ein weiteres Gerät dieser Art ist die Kreation von John Atanasoff. An seinem \glqq{}\textit{Mark I}\grqq{} arbeitete er von 1938 bis 1942 am Iowa State College. Dieser war in seiner Funktion allerdings auf Vektoraddition und Subtraktion beschränkt und nicht für universale Zwecke geeignet. Beide dieser Maschinen werden auch häufig als \glqq{}erste Computer\grqq{} betitelt.\\\\
Im Vergleich zum ENIAC und dem Mark I war die von Zuse gebaute Z1 allerdings sowohl die nützlichste, als auch die älteste Maschine. Jedoch war diese aufgrund ihrer mechanischen Bauweise zu unzuverlässig. Die größte Problematik hierbei lag darin, dass die einzelnen Bauteile extrem genau aufeinander abgestimmt sein mussten, damit keine Schäden entstanden. Sie bewies aber, dass Struktur und Konzept des Geräts funktionierten.
\subsection{Verbesserung der Z1 mithilfe der Elektronik}
Zuse begann nach alternativen Technologien zu forschen, welche zum Bau eines Rechners geeigneter waren. Sein nachfolgendes Modell, die Z2, bediente sich zwar teils noch immer an mechanischer Bauweise, nutzte für die Architektur des Rechenwerks hingegen ausschließlich elektromechanische Relais. Auch hier war die Nutzung des Binärsystems angebracht, da bei einem Relais ebenfalls nur zwei Zustände (offen und geschlossen) möglich sind. Hierdurch waren die Relais zudem mit dem mechanisch realisiertem Speicherwerk kompatibel.\\\\
Die Entwicklung der Z3 begann Konrad Zuse in 1938. Nachdem dessen Z1 gänzlich mechanisch, und die Z2 teils mithilfe von Relais realisiert wurde, bestanden sowohl das Rechen- als auch das Speicherwerk der Z3 ausschließlich aus Relais. Hierbei wurde, ebenso wie bei der Z2, an der Logikstruktur nichts verändert. Sie wurde lediglich übersetzt, wodurch die Z1 im Vergleich zur Z3 die mechanische Bauweise als einzigen Unterschied aufweist. Der Bau der Z3 wurde 1941 abgeschlossen, wodurch sie schon vier Jahre vor der Fertigstellung des ENIAC funktionsfähig war. Das Rechenwerk der Z3 nutzte 600 Relais, das Speicherwerk hingegen mit etwa 1800 Relais dreimal so viel.
\subsection{Erfolge und Probleme der Z4}
In \cite{bruderer2011konrad} wird die nächste Maschine der Reihe thematisiert. Die zwischen 1942 und 1945 konstruierte Z4 lief ebenso wie die Z3 mit Relais. Sie hatte allerdings im Vergleich zu den Vorgängermodellen und vergleichbaren amerikanischen Modellen eine deutlich geringere Anfälligkeit für Störungen. Die Z4 wurde zudem für fünf Jahre an das Institut für angewandte Mathematik an der ETH Zürich vermietet, wofür diese insgesamt $30\,000$ Franken bezahlte. Zuse stieß dabei allerdings auch auf Kritik, da die Z4 trotz der deutlichen Verbesserungen immer noch sehr wartungsintensiv war. So wurde die Z4 beispielsweise in einem Brief an Zuse vom 18. Juni 1951 von Eduard Stiefel, Gründer des Instituts für angewandte Mathematik an der ETH Züruch, als \glqq{}absolut nicht [betriebsbereit]\grqq{} bezeichnet. Im Brief heißt es auch, dass es zwei Wochen dauerte, bis es erstmals gelang, eine kleine Rechnung durchzuführen. Max Engeli bediente als Nachtoperator oftmals die Z4. Er schrieb über sie, dass er oftmals den Speicher der Z4 reparieren musste, da dort oft Knöpfe hineinfielen.\\\\
Dennoch konnte die Maschine nach zwei Wochen in Betrieb genommen werden. Sie war ikonisch für ihre Geräuschkulisse, welche aufgrund der elektromechanischen Relais entstand. Laut Erzählungen konnte man sogar mit etwas Übung erhören, welche Art der Rechenoperation das Gerät momentan ausführt. Deshalb schrieb Zuse in seiner Autobiografie, dass \glqq{}[...] das verschlafene Zürich durch die ratternde Z4 ein, wenn auch bescheidenes, Nachtleben\grqq{} bekam. Sie wurde von der ETH Zürich unter anderem genutzt, um beispielsweise lineare Differenzialgleichungen zu lösen, Spannungen in Talsperren zu berechnen oder sogar Untersuchungen in der Quantenmechanik vorzunehmen. Eine solche Berechnung durchzuführen dauerte oftmals etwa 100 Stunden. Zuse wurde schließlich am 23. November 1991 von der ETH Zürich die Ehrendoktorwürde verliehen.
\subsection{ERMETH - Der hauseigene Rechner der ETH Zürich}
Wie in \cite{bruderer2011konrad} beschrieben mietete die ETH Zürich die Z4 lediglich als Übergangslösung, da es zu der Zeit keine programmierbaren Rechner zu kaufen gab. Eduard Stiefel besuchte vom Oktober 1948 bis März 1949 die vereinigten Staaten. Dort informierte er sich über den Forschungsstand der Rechenmaschinen. Im Jahr 1949 eigneten sich zwei seiner Mitarbeiter Wissen über den Bau von Rechenmaschinen an. Hierfür bereisten sie ebenfalls die USA.\\\\
Schließlich wurde von 1953 bis 1956 die ERMETH (elektronische Rechenmaschine der ETH) erbaut. Nicht zuletzt waren am Bau Eduard Stiefel und Heinz Rutishauser beteiligt. Letzterer war Professor an der ETH und zudem einer der Pioniere im Gebiet der Computerwissenschaften.\\\\
Die ERMETH wies im Vergleich zur Z4 einige Besonderheiten auf. Während die Z1 bis Z4 im Binärsystem rechneten, arbeitete die ERMETH mit dem uns Menschen bekannten Dezimalsystem. Zudem wurde beim Bau auf Elektronenröhren zurückgegriffen, statt auf Relais. Die ERMETH wurde von 1956 bis 1963 an der ETH Zürich betrieben.