Aus heutiger Sicht ist ein Leben ohne Technologie kaum vorstellbar. Nahezu jeder besitzt ein eigenes Smartphone und/oder PC. An deutschen Schulen werden zunehmend digitale Medien für Unterrichtszwecke verwendet. Und mithilfe von \glqq{}Smart-Home\grqq{} Systemen lassen sich beispielsweise Beleuchtung und Heizungen bequem mit dem Smartphone steuern.\\\\
Diese Systeme beruhen alle auf digitaler Technik. Selbst kleine Smart-Home Geräte funktionieren meist mithilfe von Mikroprozessoren. Diese ähneln im Aufbau und der Funktionsweise sehr den konventionellen Computern. Sie sind jedoch wesentlich kleiner und dadurch, vor allem in Hinsicht auf die Rechenleistung, beschränkter. Jedoch arbeiten auch diese Systeme nach denselben Prinzipien wie größere Geräte: Dem Durchführen mathematischer Berechnungen.\\\\
Ein möglicher Schaltkreis, welcher hierbei für die Addition als einfache Rechenoperation verwendet werden kann, ist der \glqq{}Binäraddierer\grqq{}, welcher im Folgenden thematisiert wird.